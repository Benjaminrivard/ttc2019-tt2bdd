\documentclass[a4paper]{scrartcl}

\usepackage[T1]{fontenc}
\usepackage[utf8]{inputenc}

\title{The TTC 2019 TT2BDD Case}
\author{
  Antonio García-Domínguez\\
  Aston University\\
  B4 7ET, Birmingham, United Kingdom\\
  a.garcia-dominguez@aston.ac.uk
}

\begin{document}

\maketitle

\begin{abstract}
  Model transformation tools have reached a considerable level of maturity in
  the core features, and are currently developing in many directions. Some tools
  are focusing on providing higher performance for large models or complex
  transformations. Others focus on bidirectionality, visualisation,
  traceability, or verifiability, among other research directions. Whereas past
  cases in TTC have focused on specific research directions, this case study
  presents a well-known simple transformation and welcomes researchers to apply
  their research to it. The aim of this case is to serve as a showcase of the
  various directions that model transformation research is going towards at the
  moment.
\end{abstract}

\section{Introduction}

Past editions of the Transformation Tool Contest have focused on a variety of
topics:
\begin{itemize}
\item In 2018, the Quality-based Software Selection and Hardware-Mapping case
  discussed optimisation-oriented model transformations (with a combination of
  performance and solution quality). The Social Media Live Case considered
  performance in updating model views as models changed (with a strong
  preference for approaches supporting incrementality).

\item In 2017, the Smart Grid case focused on incrementality, the Families to
  Persons case discussed bidirectional transformations, State Elimination
  focused on performance and the live case on Transformation Reuse discussed
  mechanisms to share complex logic across multiple versions of a
  transformation.

\item In 2016, optimisation-oriented model transformations were discussed in
  considerable breadth through the Class Responsibility Assignment case, and an
  alternative dataflow-based notation for model transformation was evaluated in
  the live case study.
\end{itemize}

While these were notable examples of realistic transformations, they were
narrowly focused on a specific topic,and their inherent complexity discouraged
some attendees from trying their hand with their own research agenda on the
transformation.

In this case, we propose a broader contest that welcomes all active lines of
work on model transformation, based on a simpler, well-known transformation from
the ATL Zoo\footnote{\url{https://www.eclipse.org/atl/atlTransformations/}}: the
TT2BDD (Truth Tables to Binary Decision Diagrams) example transformation.
Striving for raw performance is an option, but the case welcomes approaches that
focus on other attributes of interest to a high-quality model transformation.
This includes attributes such as verifiability, traceability, bidirectionality,
or understandability, but solution providers are welcome to propose their own
attributes of interest.

The rest of the document is structured as follows:
Section~\ref{sec:transf-descr} describes the TT2BDD transformation.
Section~\ref{sec:task-suggestions} suggests several tasks of interest that could
be tackled in a solution (authors are free to propose their own tasks of
interest). Section~\ref{sec:benchmark-framework} mentions the benchmark
framework for those solutions that focus on raw performance. Finally,
Section~\ref{sec:evaluation} mentions an outline of the initial audience-based
evaluation across all solutions, and the approach that will be followed to
derive additional prizes depending on the attributes targeted by the solutions.

\section{Transformation Description}
\label{sec:transf-descr}

Short intro + structure.

\subsection{Input Metamodel: Truth Tables}
\label{sec:input-metam-truth}

Explain the TT metamodel, with diagram + example.

\subsection{Output Metamodel: Binary Decision Trees}
\label{sec:outp-metam-binary}

Explain the BDD metamodel, with diagram + example.

\subsection{Process Outline}
\label{sec:process-outline}

Explain in broad terms what the tx should do (based on ATL outline).

\section{Task Suggestions}
\label{sec:task-suggestions}

Bidirectionality and performance are two obvious ones, but this section should
make it clear that people are free to do what they want.

\section{Benchmark Framework}
\label{sec:benchmark-framework}

Mention that this is only required if focusing on performance.

\subsection{Solution requirements}
\label{sec:solut-requ}

\subsection{Running the benchmark}
\label{sec:running-benchmark}

\section{Evaluation}
\label{sec:evaluation}

Since this is an open problem, the evaluation will also depend on the submitted
solutions. There will be an audience award, where the audience will be given a
limited number of points that they can distribute (e.g. 2 per attendee). If
multiple solutions aim at the same research priorities (e.g. bidirectionality /
incremental change propagation / raw performance), a dedicated category will be
created for it.

Correctness is important, though - there is a validator that checks whether a
BDD matches an input TT model. Essentially, it runs the BDD through all the
inputs in the various rows and checks the same result is produced for each
output port.

\end{document}
