\documentclass[a4paper]{scrarticle}

\usepackage[T1]{fontenc}
\usepackage[utf8]{inputenc}

\title{The TTC 2019 TT2BDD Case}
\author{
  Antonio García-Domínguez\\
  Aston University\\
  B4 7ET, Birmingham, United Kingdom\\
  a.garcia-dominguez@aston.ac.uk
}

\begin{document}

\maketitle

\begin{abstract}
  Model transformation tools have reached a considerable level of maturity in
  the core features, and are currently developing in many directions. Some tools
  are focusing on providing higher performance for large models or complex
  transformations. Others focus on bidirectionality, visualisation,
  traceability, or verifiability, among other research directions. Whereas past
  cases in TTC have focused on specific research directions, this case study
  presents a well-known simple transformation and welcomes researchers to apply
  their research to it. The aim of this case is to serve as a showcase of the
  various directions that model transformation research is going towards at the
  moment.
\end{abstract}

\section{Introduction}

\section{Case Description}

Short intro + structure.

\subsection{Input Metamodel: Truth Tables}

Explain the TT metamodel, with diagram + example.

\subsection{Output Metamodel: Binary Decision Trees}

Explain the BDD metamodel, with diagram + example.

\subsection{Transformation Outline}

Explain in broad terms what the tx should do (based on ATL outline).

\section{Task Suggestions}

Bidirectionality and performance are two obvious ones, but this section should
make it clear that people are free to do what they want.

\section{Benchmark Framework}

Mention that this is only required if focusing on performance.

\subsection{Solution requirements}

\subsection{Running the benchmark}

\section{Evaluation}

Since this is an open problem, the evaluation will also depend on the submitted
solutions. There will be an audience award, where the audience will be given a
limited number of points that they can distribute (e.g. 2 per attendee). If
multiple solutions aim at the same research priorities (e.g. bidirectionality /
incremental change propagation / raw performance), a dedicated category will be
created for it.

Correctness is important, though - there is a validator that checks whether a
BDD matches an input TT model. Essentially, it runs the BDD through all the
inputs in the various rows and checks the same result is produced for each
output port.

\end{document}
